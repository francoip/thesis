\chapter{Conclusion} \label{chap:conclusion}

  \gls{gsm} and \gls{gprs} are legacy systems which are widely used and
  will probably stay relevant for a long time, but their security is
  outdated. Several projects emerged along the years to analyze it, and
  \proj{OsmocomBB} is one of them. By implementing an \gls{ms} side
  \gls{gsm} protocol stack running on a Calypso based platform, it
  allows an in depth control of the mobile phone side of the network. 

  Two types of attacks made possible by the \proj{OsmocomBB} project
  were analyzed in this thesis, with a focus on the \gls{gsm} system:
  eavesdropping attacks and \gls{dos} attacks. Several steps of the
  first one, and most of the second ones were implemented. While the
  eavesdropping attack is technically complicated and has only been
  demonstrated publicly by \name{Sylvain Munaut}, the \gls{dos} attacks
  are simpler to implement by modifying normal phones procedures and
  functions. An investigation on the feasibility of these two attacks
  was conducted on Norwegian networks and concluded that, while the
  eavesdropping attack would probably not be successful due to the
  failure of its many
  assumptions, the \gls{dos} attacks were difficult to prevent.

  This shows that the goals defined in~\Sref{sec:pb} were mostly
  fulfilled. Of course, improvements are possible and much work
  could still be done on the topic of \gls{gsm} and \gls{gprs}
  security. This chapter offers future work ideas for extending the
  results of this thesis.

  \Cref{chap:protocol_stack_implementation}, describing the protocol
  stack implementation \proj{OsmocomBB} could be improved by describing
  more functionalities of the project. This chapter was meant as a
  guide to the source code which would make it easier for newcomers to
  understand it and contribute to the project. An entire thesis could
  probably be written on that topic, providing an in depth analysis of
  the software and giving a good description of its architecture. It
  would also be interesting to offer more links between the
  specifications and the source code.

  \Cref{chap:eavesdropping_attacks} could obviously be improved as well.
  No public implementation of the eavesdropping attack detailed in that
  chapter has ever been provided. Of course, it might be for the best
  since the consequences are important. Still, implementing more steps
  of this attack and discussing it in more details should be interesting
  and provide a nice contribution to the field. The \gls{dos} attacks
  described in \Cref{chap:dos_attacks} could be completed by extensive
  measurements, while keeping in mind that the implementation provided
  here were not designed for efficiency, but for pedagogical purposes.
  The discussion on the feasibility of these attacks, offered in
  \Cref{chap:security_configuration_of_norwegian_operators} would
  benefit from a wider set of data gathered in the whole country. 

  Finally, a lot of work can still be done on the \proj{OsmocomBB}
  project. For example, \gls{gprs} support is not provided yet, and
  might offer new insight into the security of this network. Global
  improvement of the project could only benefit the research in the
  field of \gls{gsm} and \gls{gprs} security. Hopefully, it will
  continue to grow, and incentivise the operators, but also the equipment
  manufacturers to increase the security of their products.
