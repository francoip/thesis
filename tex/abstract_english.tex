\pagestyle{empty}
\begin{abstract}


This thesis analyzes the security of Norwegian GSM and GPRS networks
using the OsmocomBB project and considering two types of attacks: an
eavesdropping attack, and a set of Denial-of-Service attacks. OsmocomBB
aims to create a free and open source GSM baseband software
implementation. Doing so, it enhances cheap and accessible compatible
phones by giving access to their inner workings. The eavesdropping
attack was presented at the 27th Chaos Communication Congress by Silvain
Munaut and Karsten Nohl. The set of Denial-of-Service attacks is
composed of: the RACH flood attack, the IMSI attach flood attack, the
IMSI detach attack, and an attack based on race conditions in the paging
process.

This thesis first introduces the projects which are related to
\proj{OsmocomBB}, and offers a guide into the GSM system by linking the
specifications with the \proj{OsmocomBB} source code. Then, it describes
the eavesdropping and Denial-of-Service attacks in details. Some steps
of the eavesdropping attack as well as the first three Denial-of-Service
attacks are implemented. Finally, the results of measurements and
experiments conducted on Norwegian networks to assess the feasibility of
the attacks are presented.

It was found that both \comp{Telenor} and \comp{Netcom} seem protected
from the eavesdropping attack, since they renegotiate a TMSI for each
service and provide the A5/3 encryption algorithm. The IMSI detach
attack was the only Denial-of-Service attack tested on these networks,
since it targets a single user. It is effective on the Telenor network,
but not on the Netcom network. The other Denial-of-Service attacks are
probably effective, but were not tested since they could damage the
networks.

\end{abstract}
